\documentclass[10pt]{article} 
\usepackage[margin=0.5in]{geometry}
\usepackage{setspace} 
\usepackage{graphicx} 
\usepackage{caption}
\usepackage{multicol}
\usepackage{subcaption} 
\usepackage{array} 
\usepackage{booktabs}
\usepackage{listings} 
\usepackage{float}
\usepackage{enumitem}

% 1.5 line spacing
\setstretch{1.5}

\title{Design Process of Room Temperature Monitor} \author{Giacomo Cappelletto\\ BU ID: U91023753} % leave empty to omit date
\date{}
\begin{document} \maketitle

\section{Summary}
% Quarter- to half-page.  Briefly state what you did, how you did it, main
% outcomes & significance.  Do NOT include background, motivation, or refer to
% figures/tables.
\noindent
% \dots

\section{Introduction}

Maintaining a comfortable and stable indoor temperature
is crucial to modern building management and directly impacts both energy
consumption as well as the wellbeing of those who occupy it. Traditional thermostats regulate heating and
cooling systems based on set threshold temperatures, but commercial units can be costly
and therefore inaccessible for small-scale or experimental applications. In response to
growing concerns over energy efficiency and sustainability, it is important to
explore low-cost, modular solutions that allow fine-grained temperature
monitoring and control in real world environments.

This project addresses the need for an accessible, do-it-yourself room
temperature monitoring system by integrating a TMP36 analog temperature sensor,
a $16\times 2$ I\textsuperscript{2}C LCD, and an Arduino UNO microcontroller. By
automating the measurement and display of ambient temperature, the device can
inform control actions in the event loop (such as triggering a fan or heater),
enabling users to avoid over-adjusting the temperature in their spaces, reducing
unnecessary energy use. A $9\,$V battery, switch, green and red LEDs, and a
buzzer provide basic power, status indication, and alerts without reliance on
wall plug electricity.

This project has two main aims: first, to replicate the core functionality of a
simple thermostat using components and techniques covered in our course. Second,
to develop skills in assembling electronic hardware, writing
microcontroller code (in C++), and designing enclosures in CAD. For that part of the project
enclosure, lid, battery holder, and internal baseplate were modeled and
fabricated to store and organize all components, wired using $22$ AWG and
female-to-female jumper cables for I\textsuperscript{2}C data connections.

The purpose of this report is to document the design process, from component
selection and design to circuit assembly and firmware development, and
to evaluate the system's performance against expected objectives. By following
the design methodology outlined here, we demonstrate how simple and
easily sourceable parts can be combined into a cohesive temperature
monitoring tool.


\section{Design elements} Explain your design decisions. Address each of the
following:

\begin{enumerate}[label=\Alph*.]
	\item \textbf{List Of Used Components}
	      \begin{multicols}{2}
		      \begin{itemize}
			      \item 1 $\times$ Injection Mold ABS Enclosure (Bottom and Lid)
			      \item 1 $\times$ Transparent Acrylic Lase Cut Base Plate
			      \item 8 $\times$ Polycarbonate Flat Top Phillips Screws
			      \item 8 $\times$ Polycarbonate Bolts
			      \item 8 $\times$ Metal Round Head Phillips Screws
			      \item 8 $\times$ Plastic 0.5 mm spacers
			      \item 1 $\times$ $9 V$ Battery
			      \item 1 $\times$ PLA FDM Printed $9 V$ Battery holder
			      \item 1 $\times$ Arduino UNO microcontroller
			      \item 1 $\times$ Alphanumeric $16 \times 2$ I\textsuperscript{2}C LCD with IIC
			      \item 1 $\times$ TMP36 Analog Temperature sensor
			      \item 1 $\times$ Piezo Capsule Buzzer
			      \item 1 $\times$ Red LED
			      \item 1 $\times$ Green LED
			      \item 1 $\times$ 2 way switch
			      \item 2 $\times$ Female-Female $4''$ jumper wires
			      \item 15 $\times$ $6''$ 22 AWG Wires (White and Red)
			      \item 1 $\times$ $220 \Omega$ resistor
			      \item 1 $\times$ $1 k\Omega$ resistor
			      \item 2 $\times$ Twist Nut Caps
			      \item 2 $\times$ Spade Connectors
		      \end{itemize}
	      \end{multicols}
	\item \textbf{Precision measurements}
	      \begin{table}[H]
		      \centering
		      \caption{Relevant dimensions of major components (see Appendix A for Figures 2-8)}
		      \label{tab:dim_condensed}
		      \begin{tabular}{l c c c c c}
			      \toprule
			      Item                                               & Fig.\ ID                & W [mm] & L [mm] & H [mm] & $\O$ [mm] \\
			      \midrule
			      Full Assembly                                      & \ref{fig:full_assembly} & 123,8  & 146,1  & 63,1   & -         \\

			      Injection Mold ABS Enclosure Bottom                & \ref{fig:1-c}           & 119,0  & 146,1  & 57,4   & -         \\
			      Injection Mold ABS Enclosure Lid                   & \ref{fig:2-c}           & 119,0  & 120,1  & 5,5    & -         \\
			      Transparent Acrylic Laser-Cut Base Plate           & \ref{fig:3-c}           & 95,0   & 96,1   & 9,0    & -         \\
			      Arduino UNO Microcontroller                        & \ref{fig:4-c}                       & 53,3   & 74,9   & 15,2   & -         \\
			      9 V Battery and PLA-Printed Holder                  & \ref{fig:5-c}                       & 29,7   & 48,8   & 21,0   & -         \\
			      Alphanumeric $16\times2$ I\textsuperscript{2}C LCD & \ref{fig:6-c}                       & 36,0   & 80,0   & 22,0   & -         \\
			      TMP36 Analog Temperature Sensor                    & \ref{fig:7-c}                       & -      & -      & 14,6   & 2,5       \\
			      Piezo Capsule Buzzer                               & \ref{fig:8-c}                       & -      & -      & 21,3   & 20,0      \\
			      LEDs (Red and Green)                                & \ref{fig:8-c}                       & -      & -      & 36,5   & 3,0       \\
			      2-Way Switch                                       & \ref{fig:8-c}                       & 14,8   & 20,8   & 23,3   & -         \\
			      \bottomrule
		      \end{tabular}
	      \end{table}
	\item \textbf{CAD drawings}
	      \begin{figure}[H] \centering
		      \includegraphics[width=0.7\textwidth]{cad/c-c.jpeg}
		      \caption{CAD assembly of the prototype.}
		      \label{fig:full_assembly}
	      \end{figure}


	      \begin{center}
		      \textit{Figures 2-8 in Appendix (see Section \ref{sec:appendixA})}
	      \end{center}



	\item \textbf{Prototype photographs}\\ Top view of actual prototype, with and
	      without lid, powered on.  \begin{figure}[H] \centering
		      \begin{subfigure}[b]{0.45\textwidth}
			      % \includegraphics[width=\textwidth]{prototype_with_lid.jpg}
			      \caption{With lid} \end{subfigure} \hfill \begin{subfigure}[b]{0.45\textwidth}
			      % \includegraphics[width=\textwidth]{prototype_without_lid.jpg}
			      \caption{Without lid} \end{subfigure} \caption{Top views of the working
			      prototype.} \end{figure}

	\item \textbf{Purpose of using an Arduino board}\\ Explain role of the Arduino
	      in the circuit.

	\item \textbf{Wiring diagram and methods}\\ Show diagram and discuss soldering,
	      jumper wires, twist nut caps, spade connectors, etc.  \begin{figure}[H]
		      \centering
		      % \includegraphics[width=0.8\textwidth]{wiring_diagram.png}
		      \caption{Circuit wiring diagram.} \end{figure}

	\item \textbf{Wire gauge and resistor values}\\ Specify wire gauge used and
	      justify choice (jumper vs.\ 22 AWG).  State resistor values in series with
	      green/red LEDs, calculate operating currents using KVL.

	\item \textbf{Internal power supply}\\ Discuss why a 9 V battery, its charge
	      capacity, expected runtime, and external power options.

	\item \textbf{Arduino code}\\ Provide the code listing used in your design.
	      \begin{lstlisting}[language=C, caption={Arduino sketch for running average temperature monitoring}]
#include <Wire.h>
#include <LiquidCrystal_I2C.h>

const int TEMP_PIN = A0;
const int RED_LED_PIN = 3;
const int BUZZER_PIN = 5;
const float TEMP_THRESHOLD_C = 26.0;
const int LCD_ADDR = 0x27;
const int LCD_COLS = 16;
const int LCD_ROWS = 2;
const unsigned long READ_INTERVAL_MS = 1000;
const int BUZZER_FREQ_HZ = 1000;
const int NUM_SAMPLES = 5; // sample size (NUM_SAMPLES * READ_INTERVAL_MS) is 5s

float TempC = 0.0;
float TempF = 0.0;
bool isAlarmActive = false;
unsigned long lastReadTime = 0;
float tempSumC = 0.0;
int sampleCount = 0;

LiquidCrystal_I2C lcd(LCD_ADDR, LCD_COLS, LCD_ROWS);

void setup() {
	lcd.init();
	lcd.backlight();
	lcd.setCursor(3, 0);
	lcd.print("Starting...");

	pinMode(RED_LED_PIN, OUTPUT);
	pinMode(BUZZER_PIN, OUTPUT);

	digitalWrite(RED_LED_PIN, LOW);
	noTone(BUZZER_PIN);

	delay(1500);
	lcd.clear();
	printNormalLabels();
}

void loop() {
	unsigned long currentTime = millis();
	if (currentTime - lastReadTime >= READ_INTERVAL_MS) {
		lastReadTime = currentTime;

		int TADC = analogRead(TEMP_PIN);
		float VTEMP = 5.0 * (TADC / 1024.0);
		float newTempC = 100.0 * (VTEMP - 0.5);

		// running average filter for smooth temperature displaying
		tempSumC -= TempC;
		tempSumC += newTempC;
		TempC = tempSumC / NUM_SAMPLES;
		TempF = TempC * 9.0 / 5.0 + 32.0;

		bool shouldAlarmBeActive = (TempC > TEMP_THRESHOLD_C);

		if (shouldAlarmBeActive != isAlarmActive) {
			isAlarmActive = shouldAlarmBeActive;
			lcd.clear();

			if (isAlarmActive) {
				digitalWrite(RED_LED_PIN, HIGH);
				tone(BUZZER_PIN, BUZZER_FREQ_HZ);

				lcd.setCursor(0, 0);  // Top left
				lcd.print("!!");

				lcd.setCursor(5, 0);
				lcd.print(TempC, 1);
				lcd.print((char)223);
				lcd.print("C");

				lcd.setCursor(14, 0); // Top right (leave space)
				lcd.print("!!");

				lcd.setCursor(0, 1); // Bottom Left
				lcd.print("!!");

				lcd.setCursor(5, 1);
				lcd.print(TempF, 1);
				lcd.print((char)223);
				lcd.print("F");

				lcd.setCursor(14, 1); // Bottom Right (leave space)
				lcd.print("!!");
			} else {
				digitalWrite(RED_LED_PIN, LOW);
				noTone(BUZZER_PIN);
				printNormalLabels();
				updateNormalDisplay();
			}
		} else {
			if (!isAlarmActive) {
				updateNormalDisplay();
			}
		}
	}
}

void printNormalLabels() {
	lcd.setCursor(9, 0);
	lcd.print((char)223);
	lcd.print("C");
	lcd.setCursor(9, 1);
	lcd.print((char)223);
	lcd.print("F");
}

void updateNormalDisplay() {
	lcd.setCursor(5, 0);
	lcd.print(TempC, 1);
	if (TempC < 10.0 && TempC >= 0.0) lcd.print(" ");
	if (TempC < 0.0 && TempC > -10.0) lcd.print(" ");

	lcd.setCursor(5, 1);
	lcd.print(TempF, 1);
	if (TempF < 100.0 && TempF >= 10.0) lcd.print(" ");
	else if (TempF < 10.0 && TempF >= 0.0) lcd.print("  ");
	else if (TempF < 0.0 && TempF > -10.0) lcd.print(" ");
}
\end{lstlisting}


	\item \textbf{Prototype specifications}\\ List: \begin{itemize} \item Voltage of
		            power supply \item Operating voltage of circuit \item Total current drawn
		            (measured with DMM) \item Battery operating time \item Sensor temperature range
		            (from datasheet) \item Comfortable temperature range \item Use KVL to explain
		            resistor choices (1 k $\Omega$ for green LED, 220 $\Omega$ for red LED)
	      \end{itemize}

\end{enumerate}

\section{Evaluation of Results} \begin{itemize} \item Degree to which design
	      objectives were met \item Summary of outcomes and comparison to standard
	      thermometers \item Limitations and recommendations for future work (battery
	      life, size, weight, etc.) \item Highlight usefulness of the design \end{itemize}

\appendix

\section{Detail CAD Drawings of Components} \label{sec:appendixA}
% Detailed sketches or CAD technical drawings that are too long for the main
% body.
\noindent

\begin{figure}[H] \centering
	\includegraphics[width=0.7\textwidth]{cad/1-c.jpeg}
	\caption{Full assembly of the prototype.}
	\label{fig:1-c}
\end{figure}


\begin{figure}[H] \centering
	\includegraphics[width=0.7\textwidth]{cad/2-c.jpeg}
	\caption{Full assembly of the prototype.}
	\label{fig:2-c}
\end{figure}


\begin{figure}[H] \centering
	\includegraphics[width=0.7\textwidth]{cad/3-c.jpeg}
	\caption{Full assembly of the prototype.}
	\label{fig:3-c}
\end{figure}


\begin{figure}[H] \centering
	\includegraphics[width=0.7\textwidth]{cad/4-c.jpeg}
	\caption{Full assembly of the prototype.}
	\label{fig:4-c}
\end{figure}


\begin{figure}[H] \centering
	\includegraphics[width=0.7\textwidth]{cad/5-c.jpeg}
	\caption{Full assembly of the prototype.}
	\label{fig:5-c}
\end{figure}


\begin{figure}[H] \centering
	\includegraphics[width=0.7\textwidth]{cad/6-c.jpeg}
	\caption{Full assembly of the prototype.}
	\label{fig:6-c}
\end{figure}


\begin{figure}[H] \centering
	\includegraphics[width=0.7\textwidth]{cad/7-c.jpeg}
	\caption{Full assembly of the prototype.}
	\label{fig:7-c}
\end{figure}


\end{document}
