\documentclass[10pt]{article} \usepackage[margin=1in]{geometry}
\usepackage{setspace} \usepackage{graphicx} \usepackage{caption}
\usepackage{subcaption} \usepackage{array} \usepackage{booktabs}
\usepackage{listings} \usepackage{float}

% 1.5 line spacing
\setstretch{1.5}

\title{Design process of room temperature monitor} \author{Your Name \\ BU ID:
Your BU ID} \date{}  % leave empty to omit date

\begin{document} \maketitle

% REMINDER: Report should be no more than 5 pages.

\section{Summary}
% Quarter- to half-page.  Briefly state what you did, how you did it, main
% outcomes & significance.  Do NOT include background, motivation, or refer to
% figures/tables.
\noindent
% \dots

\section{Introduction} Maintaining a stable indoor temperature is key to both
comfort and energy efficiency. Commercial thermostats can be costly and rigid,
so this project explores a low-cost, modular alternative. We build a room
temperature monitor using a TMP36 sensor, a $16\times2$ I\textsuperscript{2}C
LCD, and an Arduino Uno, powered by a $9\,$V battery, with green and red LEDs
(with $1\,\mathrm{k}\Omega$ and $220\,\Omega$ resistors) and a buzzer for status
indication.

The aims are to mimic basic thermostat functions using components we explored
during the semeter and to gain hands-on experience in electronics assembly,
firmware development, and CAD design. We modeled and fabricated a enclosure,
lid, battery holder, and base plate in CAD (OnShape), wiring everything with
$22$ AWG and female-to-female jumper cables for the I\textsuperscript{2}C
monitor.

This report details the component selection, mechanical design, circuit
assembly, and code implementation, and evaluates the system's performance in
achieving the design objectives.

\section{Design elements} Explain your design decisions. Address each of the
following:

\begin{enumerate}
	\item \textbf{List all components used in your design}
	      electronic components, hardware, etc.  \item \textbf{Precision measurements}
	      Make a table of relevant dimensions (see example below).  
		  \begin{table}[H]
		      \centering \caption{Relevant dimensions of all components} 
			  \begin{tabular}{l l c c c c}
				\toprule
				Item & Sketch identification & W [mm] & L [mm] & H [mm] & Diameter [mm] \\
				\midrule
				ABS enclosure        &                  &        &        &        &               \\
				Arduino Board        &                  &        &        &        &               \\
				Switch               &                  &        &        &        &               \\
				LCD 2X16             &                  &        &        &        &               \\
				Buzzer               &                  &        &        &        &               \\
				Temperature sensor   &                  &        &        &        &               \\
				LED                  &                  &        &        &        &               \\
				\bottomrule
			  \end{tabular} 
			\end{table}

	\item \textbf{CAD drawings}\\ Include a screenshot of your final assembly.
	      \begin{figure}[H] \centering
		      % \includegraphics[width=0.7\textwidth]{cad_assembly.png}
		      \caption{CAD assembly of the prototype.} \end{figure}

	\item \textbf{Prototype photographs}\\ Top view of actual prototype, with and
	      without lid, powered on.  \begin{figure}[H] \centering
		      \begin{subfigure}[b]{0.45\textwidth}
			      % \includegraphics[width=\textwidth]{prototype_with_lid.jpg}
			      \caption{With lid} \end{subfigure} \hfill \begin{subfigure}[b]{0.45\textwidth}
			      % \includegraphics[width=\textwidth]{prototype_without_lid.jpg}
			      \caption{Without lid} \end{subfigure} \caption{Top views of the working
			      prototype.} \end{figure}

	\item \textbf{Purpose of using an Arduino board}\\ Explain role of the Arduino
	      in the circuit.

	\item \textbf{Wiring diagram and methods}\\ Show diagram and discuss soldering,
	      jumper wires, twist nut caps, spade connectors, etc.  \begin{figure}[H]
		      \centering
		      % \includegraphics[width=0.8\textwidth]{wiring_diagram.png}
		      \caption{Circuit wiring diagram.} \end{figure}

	\item \textbf{Wire gauge and resistor values}\\ Specify wire gauge used and
	      justify choice (jumper vs.\ 22 AWG).  State resistor values in series with
	      green/red LEDs, calculate operating currents using KVL.

	\item \textbf{Internal power supply}\\ Discuss why a 9 V battery, its charge
	      capacity, expected runtime, and external power options.

	\item \textbf{Arduino code}\\ Provide the code listing used in your design.
	      \begin{lstlisting}[language=C, caption={Arduino sketch for temperature
monitoring}]
// Example placeholder code
void setup() { Serial.begin(9600);
// initialization...
} void loop() { float temp = readTemperature(); Serial.println(temp);
// control LEDs...
delay(1000); } \end{lstlisting}


	\item \textbf{Prototype specifications}\\ List: \begin{itemize} \item Voltage of
		            power supply \item Operating voltage of circuit \item Total current drawn
		            (measured with DMM) \item Battery operating time \item Sensor temperature range
		            (from datasheet) \item Comfortable temperature range \item Use KVL to explain
		            resistor choices (1 k $\Omega$ for green LED, 220 $\Omega$ for red LED)
	      \end{itemize}

\end{enumerate}

\section{Evaluation of Results} \begin{itemize} \item Degree to which design
	      objectives were met \item Summary of outcomes and comparison to standard
	      thermometers \item Limitations and recommendations for future work (battery
	      life, size, weight, etc.) \item Highlight usefulness of the design \end{itemize}

\appendix \section{Supporting Materials}
% Detailed sketches or CAD technical drawings that are too long for the main
% body.
\noindent
% \dots

\end{document}

