\documentclass[10pt]{article}
\usepackage{amsmath,amssymb}
\usepackage[margin=1in]{geometry}
\usepackage{graphicx}
\usepackage{siunitx}
\begin{document}

\section*{B}

\begin{center}
	\includegraphics[width=0.5\textwidth]{b.jpeg}
\end{center}


\section*{C}

\begin{center}
	\includegraphics[width=0.5\textwidth]{c.jpeg}
\end{center}

\section*{D}
\begin{center}
	\includegraphics[width=0.5\textwidth]{d.jpeg}
\end{center}

\section*{E}
\begin{center}
	\includegraphics[width=0.5\textwidth]{e.jpeg}
\end{center}

\section*{F}
\[
	1.99 + 0.71 + 6.67 = 9.37 \approx 9.45
\]

The measured value of the total voltage as a sum of the components of the circuit is close to the theoretical value, but off by 0.08 volts. This is a small error, and is likely due to the tolerance of the resistors used in the circuit. The resistors have a tolerance of 5\%, which means that their actual resistance can vary by 5\% from the nominal value. This can lead to small errors in the calculated voltage drop across each resistor, and thus in the total voltage.

\section*{G}
\begin{center}
	\includegraphics[width=0.5\textwidth]{g.jpeg}
	\includegraphics[width=0.5\textwidth]{g2.jpeg}
\end{center}

\section*{H}

Assuming no resistance in the wires, and knowing that ideal diodes and Light Emitting Diodes have 0 resistance, the current should be

\[
    I = \frac{V}{R} = \frac{9.45}{470} = 0.0201 A
\]





\end{document}