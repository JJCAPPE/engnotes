\documentclass{article}
\usepackage[margin=1in]{geometry}
\usepackage[utf8]{inputenc}
\usepackage{amsmath}
\usepackage{tikz}
\usepackage{pgfplots}
\pgfplotsset{compat=1.18}
\usepackage{color}
\usepackage{amsfonts}
\usepackage{amssymb}
\usepackage{graphicx}

\title{P.S. 6}
\author{Giacomo Cappelletto}
\date{\today}

\begin{document}

\maketitle

\section*{1}

\subsection*{A}

\[
	\begin{bmatrix}
		-3 & 6   \\
		4  & 8   \\
		5  & -10
	\end{bmatrix}
	\xrightarrow{R_2=R_2+\frac{4}{3}R_1}
	\begin{bmatrix}
		-3 & 6  \\
		0  & 16 \\
		1  & -2
	\end{bmatrix}
	\xrightarrow{R_3=R_3+\frac{1}{3}R_1}
	\begin{bmatrix}
		-3 & 6  \\
		0  & 16 \\
		0  & -8
	\end{bmatrix}
	\xrightarrow{R_3=R_3+\frac{1}{2}R_2}
	\begin{bmatrix}
		-3 & 6  \\
		0  & 16 \\
		0  & 0
	\end{bmatrix}
\]

\[
	Col(A) = span \left\{\begin{bmatrix} -3 \\ 4 \\ 5 \end{bmatrix},\begin{bmatrix} 6 \\ 8 \\ -10 \end{bmatrix} \right\}
\]

\subsection*{B}

\[
	Basis(Col(A)) = \left\{ \begin{bmatrix} -3 \\ 4 \\ 5 \end{bmatrix}, \begin{bmatrix} 6 \\ 8 \\ -10 \end{bmatrix} \right\}
\]
\[
	Dim(Col(A)) = 2
\]
\(Col(A)\) will be a plane in \(\mathbb{R}^3\).

\subsection*{C}

From
\[
	Col(A) = \left\{ t_1 \begin{bmatrix} -3 \\ 4 \\ 5 \end{bmatrix} + t_2 \begin{bmatrix} 6 \\ 8 \\ -10 \end{bmatrix} \right\}
\]
Choose
\[
	t_1=3 , t_2=2
\]
Then
\[
	b_1=
	\begin{bmatrix}
		3 \\28\\-5
	\end{bmatrix}
\]
And from MATLAB

\begin{verbatim}
	c_1= [-3;4;5];
	c_2=[6;8;-10];
	
	B = c_1*3 + c_2*2;
	
	Fin = [c_1,c_2,B];
	
	disp(rref(Fin))
	\end{verbatim}

\color{black} \begin{verbatim}     
		 1     0     3
		 0     1     2
		 0     0     0
\end{verbatim} \color{black}

The system of equations is consistent because \(b1\) is a linear combination of the two columns of \(A\), and therefore another vector in the plane that is \(Col(A)\).

\subsection*{D}

Taking the cross product of the columns of \(A\) we get


\begin{verbatim}
	c_1= [-3;4;5];
	c_2=[6;8;-10];
	
	disp(cross(c_1, c_2))
	\end{verbatim}
\color{black}
\begin{verbatim}   
	-80
	0
	-48
\end{verbatim} \color{black}

And scaling down to get

\[
	b_2=
	\begin{bmatrix}
		-10 \\ 0 \\ -6
	\end{bmatrix}
\]

Which is perpendicular to the plane spanned by \(Col(A)\), and therefore not in the span of \(Col(A)\). Consequently, plugging \(b_2\) into the system of equations will yield an inconsistent system.\\
And proof from MATLAB

\begin{verbatim}
	c_1= [-3;4;5];
	c_2=[6;8;-10];

	B = cross(c_1 ,c_2);
	
	Fin = [c_1,c_2,B];

	disp(rref(Fin))
	\end{verbatim}

\color{black} \begin{verbatim}     
		 1     0     0
		 0     1     0
		 0     0     1
\end{verbatim} \color{black}

Which is an inconsistent system.

\section*{2}

\subsection*{A}

\[
	\begin{bmatrix}
		3 & 2 & 5  & -5 \\
		0 & 4 & -2 & 8  \\
		2 & 0 & 4  & -6
	\end{bmatrix}
	\xrightarrow{R_3=R_3-\frac{2}{3}R_1}
	\begin{bmatrix}
		3 & 2            & 5           & -5           \\
		0 & 4            & -2          & 8            \\
		0 & -\frac{4}{3} & \frac{2}{3} & -\frac{8}{3}
	\end{bmatrix}
	\xrightarrow{R_3=R_3+\frac{1}{3}R_2}
	\begin{bmatrix}
		3 & 2 & 5  & -5 \\
		0 & 4 & -2 & 8  \\
		0 & 0 & 0  & 0
	\end{bmatrix}
\]
\[
	\xrightarrow{
		\begin{aligned}
			R_1=\frac{1}{3}R_1 \\
			R_2=\frac{1}{4}R_2
		\end{aligned}
	}
	\begin{bmatrix}
		1 & \frac{2}{3} & \frac{5}{3}  & -\frac{5}{3} \\
		0 & 1           & \frac{-1}{2} & 2            \\
		0 & 0           & 0            & 0
	\end{bmatrix}
	\xrightarrow{R_1=R_1-\frac{2}{3}R_2}
	\begin{bmatrix}
		1 & 0 & 2            & -3 \\
		0 & 1 & \frac{-1}{2} & 2  \\
		0 & 0 & 0            & 0  \\
	\end{bmatrix}
\]

\[
	x_3=t_3 \, , \, x_4=t_4 \Rightarrow \vec{x}=
	\begin{bmatrix}
		-2t_3 + 3t_4           \\
		-\frac{1}{2}t_3 - 2t_4 \\
		t_3                    \\
		t_4                    \\
	\end{bmatrix}
	=
	t_3
	\begin{bmatrix}
		-2           \\
		-\frac{1}{2} \\
		1            \\
		0
	\end{bmatrix}
	+
	t_4
	\begin{bmatrix}
		3  \\
		-2 \\
		0  \\
		1
	\end{bmatrix}
\]
\[
	\therefore Null(B)= span \left\{
	\begin{bmatrix}
		-2           \\
		-\frac{1}{2} \\
		1            \\
		0
	\end{bmatrix},
	\begin{bmatrix}
		3  \\
		-2 \\
		0  \\
		1
	\end{bmatrix}
	\right\}
\]

\subsubsection*{B}

\[
	Basis(Null(B))=\left\{
	\begin{bmatrix}
		-2           \\
		-\frac{1}{2} \\
		1            \\
		0
	\end{bmatrix},
	\begin{bmatrix}
		3  \\
		-2 \\
		0  \\
		1
	\end{bmatrix}
	\right\}
\]
\[
	Dim(Null(B)) = 2
\]
The dimension of the nullspace of B corresponds to a plane spanned by the two vectors that are the basis of the nullspace.

\subsection*{C}

\[
	\vec{x_h} =t_3
	\begin{bmatrix}
		-2           \\
		-\frac{1}{2} \\
		1            \\
		0
	\end{bmatrix}+
	t_4
	\begin{bmatrix}
		3  \\
		-2 \\
		0  \\
		1
	\end{bmatrix}
\]
\(\vec{x_h}\) is exactly \(Null(B)\)

\subsection*{D}

\[
	t_3 = 2 \, , \, t_4 = 1
\]
\[
	\vec{x}=
	\begin{bmatrix}
		-4 \\-1\\2\\0
	\end{bmatrix}+
	\begin{bmatrix}
		3 \\-2\\0\\1
	\end{bmatrix}
	=
	\begin{bmatrix}
		-1 \\-3\\2\\1
	\end{bmatrix}
\]

\section*{3}

\subsection*{A}
The RREF of \(B\) is:
\[
\begin{bmatrix}
1 & 0 & 2 & -3 \\
0 & 1 & -\frac{1}{2} & 2 \\
0 & 0 & 0 & 0
\end{bmatrix}
\]
The non-zero rows of this matrix form a basis for the row space. Therefore, the row space of \(B\) is:
\[
\text{Row } B = \text{span} \left\{ 
\begin{bmatrix} 1 \\ 0 \\ 2 \\ -3 \end{bmatrix}, \ 
\begin{bmatrix} 0 \\ 1 \\ -\frac{1}{2} \\ 2 \end{bmatrix} 
\right\}
\]

\subsection*{B}
The row space of \(B\) is a 2-dimensional subspace of \(\mathbb{R}^4\). Geometrically, this corresponds to a \textbf{plane} in \(\mathbb{R}^4\).

\subsection*{C}
The number of pivots in matrix \(B\) is equal to the number of non-zero rows in its RREF, which is 2. The number of vectors in a basis for the row space (the dimension of the row space) is also 2.


   \section*{4} 
\begin{verbatim}
	C = [3  5  -2  -1  1;
		 2  0   2   4  2;
		-2  7  -9  -5  4];
	
	R = rref(C);
	disp(R);
	
	null_vec = null(sym(C));
	disp(null_vec);
	
	disp('C * (nullspace vector) = 0:');
	[nRows, nVec] = size(null_vec);
	for i = 1:nVec
		xn = null_vec(:, i);
		product = C * xn;
		fprintf('Result for nullspace vector %d:\n', i);
		disp(product);
	end

		 1     0     1     0    -1
		 0     1    -1     0     1
		 0     0     0     1     1
	
	[-1,  1]
	[ 1, -1]
	[ 1,  0]
	[ 0, -1]
	[ 0,  1]
	 
	C * (nullspace vector) = 0:
	Result for nullspace vector 1:
	0
	0
	0
	 
	Result for nullspace vector 2:
	0
	0
	0
	 
\end{verbatim} 






\end{document}