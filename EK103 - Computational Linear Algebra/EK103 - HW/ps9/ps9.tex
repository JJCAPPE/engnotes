\documentclass{article}
\usepackage{color}
\usepackage[margin=1in]{geometry}
\usepackage[utf8]{inputenc}
\usepackage{amsmath}
\usepackage{amsfonts}
\usepackage{amssymb}
\usepackage{graphicx} % Not used here, but often useful

% Define matrix and vector commands for convenience
\newcommand{\mat}[1]{\begin{bmatrix} #1 \end{bmatrix}}
\newcommand{\vect}[1]{\begin{bmatrix} #1 \end{bmatrix}}
\title{PS9}
\author{Giacomo Cappelletto}
\begin{document}

\maketitle

\section*{1}

\subsection*{A}

\[D = \begin{bmatrix} 2 & 0 \\ 0 & 3 \end{bmatrix}\]

\[P = \begin{bmatrix} 1 & 0 \\ 0 & 1 \end{bmatrix}\]

\[P^{-1} = \begin{bmatrix} 1 & 0 \\ 0 & 1 \end{bmatrix}\]

\subsection*{B}

\[D = \begin{bmatrix} 5 & 0 \\ 0 & -2 \end{bmatrix}\]

\[P = \begin{bmatrix} 1 & 1 \\ 1 & -6 \end{bmatrix}\]

\[P^{-1} = \begin{bmatrix} 6/7 & 1/7 \\ 1/7 & -1/7 \end{bmatrix}\]

\subsection*{C}

\[D = \begin{bmatrix} 4 & 0 \\ 0 & 5 \end{bmatrix}\]

\[P = \begin{bmatrix} 1 & 1 \\ 0 & 1 \end{bmatrix}\]

\[P^{-1} = \begin{bmatrix} 1 & -1 \\ 0 & 1 \end{bmatrix}\]

\subsection*{D}

\[D = \begin{bmatrix} 0 & 0 \\ 0 & 2 \end{bmatrix}\]

\[P = \begin{bmatrix} 1 & 1 \\ -1 & 1 \end{bmatrix}\]

\[P^{-1} = \begin{bmatrix} 1/2 & -1/2 \\ 1/2 & 1/2 \end{bmatrix}\]

\subsection*{E}


Not Diagonalizable:

$(1-\lambda)^2 = 0$, so $\lambda = 1$

\[(A - \lambda I) = \begin{bmatrix} 0 & 1 \\ 0 & 0 \end{bmatrix}\]

\[(A - \lambda I)\mathbf{v} = \mathbf{0}\]
\[\begin{bmatrix} 0 & 1 \\ 0 & 0 \end{bmatrix} \begin{bmatrix} x \\ y \end{bmatrix} = \begin{bmatrix} 0 \\ 0 \end{bmatrix}\]  This yields $y = 0$.  The eigenspace is spanned by \[\begin{bmatrix} 1 \\ 0 \end{bmatrix}\]

The dimension of the nullspace of $(A - \lambda I)$ is 1.  The algebraic multiplicity of $\lambda = 1$ is $k=2$. Since $1 < 2$, the matrix is not diagonalizable.
\subsection*{F}

\[D = \begin{bmatrix} 1 & 0 & 0 \\ 0 & 1 & 0 \\ 0 & 0 & 1 \end{bmatrix}\]

\[P = \begin{bmatrix} 1 & 0 & 0 \\ 0 & 1 & 0 \\ 0 & 0 & 1 \end{bmatrix}\]

\[P^{-1} = \begin{bmatrix} 1 & 0 & 0 \\ 0 & 1 & 0 \\ 0 & 0 & 1 \end{bmatrix}\]

\subsection*{G}

\[D = \begin{bmatrix} 1 & 0 & 0 \\ 0 & 3 & 0 \\ 0 & 0 & 3 \end{bmatrix}\]

\[P = \begin{bmatrix} 1 & 1 & 0 \\ 0 & 0 & 1 \\ -1 & 1 & 0 \end{bmatrix}\]

\[P^{-1} = \begin{bmatrix} 1/2 & 0 & -1/2 \\ 1/2 & 0 & 1/2 \\ 0 & 1 & 0 \end{bmatrix}\]

\section*{2}

\subsection*{A}

$\lambda_1 = 2, \lambda_2 = 4$

$\mathbf{v_1} = \begin{bmatrix} 1 \\ 0 \end{bmatrix}, \mathbf{v_2} = \begin{bmatrix} 1 \\ 1 \end{bmatrix}$

\[P = \begin{bmatrix} 1 & 1 \\ 0 & 1 \end{bmatrix},  D = \begin{bmatrix} 2 & 0 \\ 0 & 4 \end{bmatrix}, P^{-1} = \begin{bmatrix} 1 & -1 \\ 0 & 1 \end{bmatrix}\]

\[D^5 = \begin{bmatrix} 2^5 & 0 \\ 0 & 4^5 \end{bmatrix} = \begin{bmatrix} 32 & 0 \\ 0 & 1024 \end{bmatrix}\]

\[A^5 = PDP^{-1} =  \begin{bmatrix} 1 & 1 \\ 0 & 1 \end{bmatrix} \begin{bmatrix} 32 & 0 \\ 0 & 1024 \end{bmatrix} \begin{bmatrix} 1 & -1 \\ 0 & 1 \end{bmatrix} = \begin{bmatrix} 32 & 992 \\ 0 & 1024 \end{bmatrix}\]

\subsection*{B}

$\lambda_1 = 2, \lambda_2 = 4, \lambda_3 = 5$

$\mathbf{v_1} = \begin{bmatrix} 1 \\ 1 \\ 0 \end{bmatrix}, \mathbf{v_2} = \begin{bmatrix} 1 \\ -1 \\ 0 \end{bmatrix}, \mathbf{v_3} = \begin{bmatrix} 0 \\ 0 \\ 1 \end{bmatrix}$

\[P = \begin{bmatrix} 1 & 1 & 0 \\ 1 & -1 & 0 \\ 0 & 0 & 1 \end{bmatrix}, D = \begin{bmatrix} 2 & 0 & 0 \\ 0 & 4 & 0 \\ 0 & 0 & 5 \end{bmatrix}, P^{-1} = \begin{bmatrix} 1/2 & 1/2 & 0 \\ 1/2 & -1/2 & 0 \\ 0 & 0 & 1 \end{bmatrix}\]

\[D^2 = \begin{bmatrix} 2^2 & 0 & 0 \\ 0 & 4^2 & 0 \\ 0 & 0 & 5^2 \end{bmatrix} = \begin{bmatrix} 4 & 0 & 0 \\ 0 & 16 & 0 \\ 0 & 0 & 25 \end{bmatrix}\]

\[A^2 = PDP^{-1} = \begin{bmatrix} 1 & 1 & 0 \\ 1 & -1 & 0 \\ 0 & 0 & 1 \end{bmatrix} \begin{bmatrix} 4 & 0 & 0 \\ 0 & 16 & 0 \\ 0 & 0 & 25 \end{bmatrix} \begin{bmatrix} 1/2 & 1/2 & 0 \\ 1/2 & -1/2 & 0 \\ 0 & 0 & 1 \end{bmatrix} = \begin{bmatrix} 10 & -6 & 0 \\ -6 & 10 & 0 \\ 0 & 0 & 25 \end{bmatrix}\]

\section*{3}

\subsection*{A}
Since the principal stresses are the eigenvalues of \(T\), then
\[
	(\lambda - 15)^2 = 25
\]
\[
	\lambda - 15 = \pm 5
\]
\[
	\sigma_1 = 10, \sigma_2 = 20
\]

\subsection*{B}
The eigenvector for \(\sigma_1\) is
\[
	\begin{bmatrix}
		5  & -5 \\
		-5 & 5
	\end{bmatrix}
	\rightarrow
	\begin{bmatrix}
		1 & -1 \\
		0 & 0
	\end{bmatrix}
	\rightarrow
	\left \{ t \begin{bmatrix}
		1 \\
		1
	\end{bmatrix}: t \in \mathbb{R}
	\right \}
\]
\[
	\vec{v_1}= \begin{bmatrix}
		1 \\
		1
	\end{bmatrix}
\]

and for \(\sigma_2\) is
\[
	\begin{bmatrix}
		-5 & -5 \\
		-5 & -5
	\end{bmatrix}
	\rightarrow
	\begin{bmatrix}
		1 & 1 \\
		0 & 0
	\end{bmatrix}
	\rightarrow
	\left \{ t \begin{bmatrix}
		1 \\
		-1
	\end{bmatrix} : t \in \mathbb{R}
	\right \}
\]
\[
	\vec{v_2}= \begin{bmatrix}
		1 \\
		-1
	\end{bmatrix}
\]

\subsection*{C}
\[
	\sigma_{vM} = \frac{\sqrt{2}}{2} \sqrt{(10-20)^2 + 10^2 + 20^2} = 10 \sqrt{3} \approx 17.32
\]
\[
	17.32 > 17 \therefore \text{ Breaks}
\]


\end{document}