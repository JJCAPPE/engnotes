\documentclass{article}
\usepackage[margin=1in]{geometry}
\usepackage[utf8]{inputenc}
\usepackage{amsmath}
\usepackage{tikz}
\usepackage{pgfplots}
\pgfplotsset{compat=1.18}
\usepackage{color}
\usepackage{amsfonts}
\usepackage{amssymb}
\usepackage{graphicx}

\title{P.S. 5}
\author{Giacomo Cappelletto}
\date{\today}

\begin{document}

\maketitle

\section*{1)}

\subsection*{A)}

\[
	\begin{bmatrix}
		1 & 3  \\
		2 & -5
	\end{bmatrix}
	\xrightarrow{R_2 = R_2 - 2R_1}
	\begin{bmatrix}
		1 & 3   \\
		0 & -11
	\end{bmatrix}
\]
Linearly Independent, as number of columns is equal to number of pivots
\subsection*{B)}

\[
	\begin{bmatrix}
		1  & -2 \\
		-3 & 6
	\end{bmatrix}
	\xrightarrow{R_2 = R_2 + 3R_1}
	\begin{bmatrix}
		1 & -2 \\
		0 & 0
	\end{bmatrix}
\]
Linearly Dependent because \(\vec{v} = \vec{u} \cdot -2\)

\subsection*{C)}
\[
	\begin{bmatrix}
		1  & 4  \\
		2  & 5  \\
		-3 & -6
	\end{bmatrix}
	\xrightarrow{
		\begin{aligned}
			R_2 & = R_2 - 2R_1 \\
			R_3 & = R_3 + 3R_1
		\end{aligned}
	}
	\begin{bmatrix}
		1 & 4  \\
		0 & -3 \\
		0 & 6
	\end{bmatrix}
\]
Linearly Independent because number of columns is equal to number of pivots (and therefore only solution to \(A\vec{x}=\vec{0}\) is the trivial one)


\subsection*{D)}

\[
	\begin{bmatrix}
		2  & 3   \\
		4  & 6   \\
		-8 & -12
	\end{bmatrix}
	\xrightarrow{
		\begin{aligned}
			R_2 & = R_2 - 2R_1 \\
			R_3 & = R_3 + 4R_1
		\end{aligned}
	}
	\begin{bmatrix}
		2 & 3 \\
		0 & 0 \\
		0 & 0
	\end{bmatrix}
\]

Linearly Dependent because \(\vec{v} = \vec{u} \cdot \frac{3}{2}\)

\section*{2)}

\subsection*{A)}

\[
	\begin{bmatrix}
		1 & 2 & 1 \\
		2 & 5 & 5 \\
		5 & 1 & 2
	\end{bmatrix}
	\xrightarrow{
		\begin{aligned}
			R_2 = R_2 - 2R_1 \\
			R_3 = R_3 - 5R_1 \\
		\end{aligned}
	}
	\begin{bmatrix}
		1 & 2  & 1  \\
		0 & 1  & 3  \\
		0 & -9 & -3
	\end{bmatrix}
	\xrightarrow{R_3 = R_3 + 9R_2}
	\begin{bmatrix}
		1 & 2 & 1  \\
		0 & 1 & 3  \\
		0 & 0 & 24
	\end{bmatrix}
\]

Linearly Independent because number of columns is equal to number of pivots (and therefore only solution to \(A\vec{x}=\vec{0}\) is the trivial one)

\subsection*{B)}

\[
	\begin{bmatrix}
		1 & 0 & 1 \\
		2 & 0 & 5 \\
		3 & 0 & 6
	\end{bmatrix}
	\xrightarrow{
		\begin{aligned}
			R_2 = R_2 - 2R_1 \\
			R_3 = R_3 - 3R_1
		\end{aligned}
	}
	\begin{bmatrix}
		1 & 0 & 1 \\
		0 & 0 & 3 \\
		0 & 0 & 3
	\end{bmatrix}
	\xrightarrow{
		R_3=R_3-R_2
	}
	\begin{bmatrix}
		1 & 0 & 1 \\
		0 & 0 & 3 \\
		0 & 0 & 0
	\end{bmatrix}
\]

Linearly Dependent due to the presence of the column \(\vec{0}\), which immediately makes the set dependent, since now the number of pivots cannot exceed 2, 1 less than number of columns

\subsection*{C)}

\[
	\begin{bmatrix}
		1 & 1 & 2 & 3 \\
		2 & 3 & 5 & 1 \\
		5 & 1 & 7 & 4
	\end{bmatrix}
	\xrightarrow{
		\begin{aligned}
			R_2 = R_2 - 2R_1 \\
			R_3 = R_3 - 5R_1
		\end{aligned}
	}
	\begin{bmatrix}
		1 & 1  & 2  & 3   \\
		0 & 1  & 1  & -5  \\
		0 & -4 & -3 & -11
	\end{bmatrix}
	\xrightarrow{
		R_3 = R_3 + 4R_2
	}
	\begin{bmatrix}
		1 & 1 & 2 & 3   \\
		0 & 1 & 1 & -5  \\
		0 & 0 & 1 & -31
	\end{bmatrix}
\]

Linearly Dependent because of free variable in column 4

\section*{3)}

\[
	C^{-1}=
	\frac{1}{-2}
	\begin{bmatrix}
		5  & -3 \\
		-4 & 2
	\end{bmatrix}
	=
	\begin{bmatrix}
		-\frac{5}{2} & \frac{3}{2} \\
		2            & -1
	\end{bmatrix}
\]

\[
	\det(D) = 0 \Longrightarrow  D^{-1} \, DNE
\]

\section*{4)}

\[
	A \times B =
	\begin{bmatrix}
		1 & 0  & 2 \\
		2 & -1 & 3 \\
		4 & 1  & 8
	\end{bmatrix}
	\times
	\begin{bmatrix}
		-11 & 2  & 2  \\
		-4  & 0  & 1  \\
		6   & -1 & -1
	\end{bmatrix}
	=
	\begin{bmatrix}
		1 & 0 & 0 \\
		0 & 1 & 0 \\
		0 & 0 & 1
	\end{bmatrix}
\]

\section*{5)}

\[
	\begin{bmatrix}
		1 & 0  & 2 & 1 & 0 & 0 \\
		2 & -1 & 3 & 0 & 1 & 0 \\
		4 & 1  & 8 & 0 & 0 & 1
	\end{bmatrix}
	\xrightarrow{
		\begin{aligned}
			R_2 = R_2 - 2R_1 \\
			R_3 = R_3 - 4R_1
		\end{aligned}
	}
	\begin{bmatrix}
		1 & 0  & 2  & 1  & 0 & 0 \\
		0 & -1 & -1 & -2 & 1 & 0 \\
		0 & 1  & 0  & -4 & 0 & 1
	\end{bmatrix}
	\xrightarrow{R_3=R_3+R_2}
	\begin{bmatrix}
		1 & 0  & 2  & 1  & 0 & 0 \\
		0 & -1 & -1 & -2 & 1 & 0 \\
		0 & 0  & -1 & -6 & 1 & 1
	\end{bmatrix}
\]
\[
	\xrightarrow{
		\begin{aligned}
			R_2 = -R_2 \\
			R_3 = -R_3 - 5R_1
		\end{aligned}
	}
	\begin{bmatrix}
		1 & 0 & 2 & 1 & 0  & 0  \\
		0 & 1 & 1 & 2 & -1 & 0  \\
		0 & 0 & 1 & 6 & -1 & -1
	\end{bmatrix}
	\xrightarrow{
		\begin{aligned}
			R_1 = R_1 - 2R_3 \\
			R_2 = R_2 - R_3
		\end{aligned}
	}
	\begin{bmatrix}
		1 & 0 & 0 & -11 & 2  & 2  \\
		0 & 1 & 0 & -4  & 0  & 1  \\
		0 & 0 & 1 & 6   & -1 & -1
	\end{bmatrix}
\]
\[
	\therefore A^{-1} =
	\begin{bmatrix}
		-11 & 2  & 2  \\
		-4  & 0  & 1  \\
		6   & -1 & -1
	\end{bmatrix}
\]

\end{document}