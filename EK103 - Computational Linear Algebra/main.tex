\documentclass{report}

\input{preamble}
\input{macros}
\input{letterfonts}

\usepackage{tikz}
\usepackage{tikz-3dplot}
\usepackage{amsmath}
\usepackage{pgfplots}
\usepackage{smartdiagram}
\usesmartdiagramlibrary{additions}
\usepackage{xcolor}
\usepackage{forest}
\usepgfplotslibrary{colormaps}
\usepgfplotslibrary{groupplots}
\usepgfplotslibrary{polar}
\pgfplotsset{compat=newest}
\tikzset{>=latex}
\usepackage{siunitx}

\title{\Huge{Computational Linear Algebra}\\EK103}
\author{\huge{Giacomo Cappelletto}}
\date{<DATE>}

\begin{document}


\maketitle
\newpage
\pdfbookmark[section]{\contentsname}{toc}
\tableofcontents
\pagebreak

\chapter{Basics}

\section{Vectors, Norms and Products}

\nt{
	Let us consider two vectors in \(\mathbb{R}^3\):
	\[
		u =
		\begin{pmatrix}
			1 \\
			1 \\
			1
		\end{pmatrix}
		\quad \text{and} \quad
		v =
		\begin{pmatrix}
			1  \\
			-1 \\
			1
		\end{pmatrix}.
	\]
	We wish to compute their magnitudes (norms and norm-squared), the angle between them, and the plane that they span. These methods are directly applicable to computational tools such as MATLAB.
}

\dfn{Norm of a Vector}{
	For a vector \(x = (x_1, x_2, \dots, x_n)\in \mathbb{R}^n\), its norm is
	\[
		\|x\| \;=\; \sqrt{x_1^2 + x_2^2 + \cdots + x_n^2}.
	\]
	In many programming languages (including MATLAB), this is computed via \texttt{norm(x)}, while the square of the norm is \(\|x\|^2 = x \cdot x = x_1^2 + \cdots + x_n^2\).
}

\ex{Norms and Norm-Squared of \(\,u\) and \(\,v\)}{
	\[
		\|u\| \;=\;\sqrt{1^2 + 1^2 + 1^2} \;=\;\sqrt{3},
		\quad
		\|v\| \;=\;\sqrt{1^2 + (-1)^2 + 1^2} \;=\;\sqrt{3}.
	\]
	Thus, both vectors have the same magnitude \(\sqrt{3}\). Their squared norms are
	\[
		\|u\|^2 \;=\;3,
		\quad
		\|v\|^2 \;=\;3.
	\]
	In MATLAB notation, one could write:
	\begin{itemize}
		\item \texttt{norm(u)} or \texttt{norm(u,2)} for the norm of \(u\).
		\item \texttt{dot(u,u)} or \texttt{norm(u)\^{}2} for \(\|u\|^2\).
	\end{itemize}
}


\dfn{Angle Between Two Vectors}{
	The angle \(\theta\) between two nonzero vectors \(u\) and \(v\) in \(\mathbb{R}^n\) is given by
	\[
		\theta \;=\; \arccos \Bigl(\frac{u \cdot v}{\|u\|\|v\|}\Bigr).
	\]
}

\ex{Angle Between \(\,u\) and \(\,v\)}{
	First, compute the dot product:
	\[
		u \cdot v
		\;=\;
		(1)(1) + (1)(-1) + (1)(1)
		\;=\;
		1 - 1 + 1
		\;=\;
		1.
	\]
	Hence,
	\[
		\theta
		\;=\;
		\arccos \Bigl(\frac{u \cdot v}{\|u\|\|v\|}\Bigr)
		\;=\;
		\arccos\Bigl(\frac{1}{\sqrt{3}\,\sqrt{3}}\Bigr)
		\;=\;
		\arccos\Bigl(\tfrac{1}{3}\Bigr).
	\]
	In MATLAB, one could write:
	\[
		\texttt{theta = acos(dot(u,v)/(norm(u)*norm(v)));}
	\]
}

\dfn{Plane Spanned by Two Vectors}{
	The plane containing vectors \(u\) and \(v\) and passing through the origin is given by
	\[
		\{\;\alpha\,u + \beta\,v \;\mid\; \alpha, \beta \in \mathbb{R}\}.
	\]
	An equivalent description is all points \(x\in \mathbb{R}^3\) such that \(x \cdot (u \times v) = 0\).
}

\ex{Plane Containing \(\,u\) and \(\,v\)}{
	\begin{itemize}
		\item
		      \emph{Span form:}
		      \[
			      \text{Plane} = \bigl\{\,\alpha \begin{pmatrix}1\\1\\1\end{pmatrix}
			      \;+\;\beta \begin{pmatrix}1\\-1\\1\end{pmatrix}
			      \;\mid\; \alpha,\beta \in \mathbb{R}\bigr\}.
		      \]
		\item
		      \emph{Normal form:}
		      The cross product
		      \[
			      u \times v
			      =
			      \begin{vmatrix}
				      \mathbf{i} & \mathbf{j} & \mathbf{k} \\
				      1          & 1          & 1          \\
				      1          & -1         & 1
			      \end{vmatrix}
			      =
			      (2,\,0,\,-2).
		      \]
		      Hence, the plane also can be described by the set of points \(x = (x_1,x_2,x_3)\) for which
		      \[
			      (2,\,0,\,-2)\cdot (x_1,x_2,x_3) = 0
			      \quad\Longrightarrow\quad
			      2\,x_1 - 2\,x_3 = 0
			      \quad\Longrightarrow\quad
			      x_1 = x_3.
		      \]
	\end{itemize}
	In many computational environments, one simply keeps the span form or uses a symbolic package to compute the cross product and normal equation.
}

\nt{
	In summary, for vectors \(u\) and \(v\):
	\begin{itemize}
		\item \(\|u\|\) and \(\|v\|\) each equal \(\sqrt{3}\).
		\item \(\|u\|^2 = \|v\|^2 = 3\).
		\item The angle between them is \(\arccos\bigl(\tfrac{1}{3}\bigr)\).
		\item The plane is spanned by \(\{u,v\}\), or equivalently described by the normal vector \(u \times v\).
	\end{itemize}
	All these computations can be done in a straightforward manner in a software package such as MATLAB, using \texttt{dot}, \texttt{norm}, \texttt{acos}, and \texttt{cross}.
}

\end{document}