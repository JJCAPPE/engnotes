\documentclass[10pt]{article}
\usepackage{amsmath,amssymb}
\usepackage[margin=1in]{geometry}
\begin{document}

\section*{1}
\[
	\text{Blue} = 6,\quad
	\text{Grey} = 8,\quad
	\text{Brown multiplier} = 10^1,\quad
	\text{Gold tolerance} = \pm 5\%.
\]

\[
	68 \times 10^1 = 680 \,\Omega \quad (\pm 5\%).
\]

\section*{2}


\[
	R = 0.05 \,\Omega/\text{m}, \quad
	\text{(22\,AWG conductor)}.
\]
22\,AWG wire is $0.6438\,\text{mm}$ in diameter, so radius is $r = 0.3219\,\text{mm}$. The cross-sectional area $A$ is:
\[
	A = \pi r^2 = \pi (0.3219\,\text{mm})^2
	\approx 0.325\,\text{mm}^2.
\]
\[
	A \approx 0.325 \times 10^{-6}\,\text{m}^2 = 3.25 \times 10^{-7}\,\text{m}^2.
\]
\[
	R = \frac{\rho \, L}{A}.
\]
\[
	\rho = \frac{R \, A}{L} = 0.05 \times 3.25 \times 10^{-7} \,\Omega \cdot \text{m}
	\approx 1.63 \times 10^{-8}\,\Omega\cdot\text{m}.
\]
This resistivity is  close to the resistivity of copper ($ 1.68 \times 10^{-8}\,\Omega\cdot\text{m}$). Therefore, the material is most likely copper.

\section*{3}

\[
	R = 2.6\,\Omega, \quad L = 1000\,\text{m}, \quad
	\rho_{\text{Cu}} = 1.72 \times 10^{-8}\,\Omega \cdot \text{m}.
\]
\[
	R = \frac{\rho L}{A}
	\quad \Longrightarrow \quad
	A = \frac{\rho L}{R}.
\]
\[
	A = \frac{(1.72 \times 10^{-8}\,\Omega\cdot\text{m})(1000\,\text{m})}{2.6\,\Omega}
	= \frac{1.72 \times 10^{-5}}{2.6}\,\text{m}^2
	\approx 6.615 \times 10^{-6}\,\text{m}^2.
\]
\[
	A = \pi \left(\frac{d}{2}\right)^2
	\quad \Longrightarrow \quad
	d = 2 \sqrt{\frac{A}{\pi}}.
\]
\[
	d = 2 \sqrt{\frac{6.615 \times 10^{-6}}{\pi}}
	\approx 2 \times 1.45 \times 10^{-3}
	\approx 2.9 \,\text{mm}.
\]
From the AWG wire tables, a diameter of  $2.9\,\text{mm}$ corresponds to 9 AWG. Therefore, the wire is probably 9 AWG

\section*{4}

\begin{enumerate}
	\item A broken or cut wire in a circuit.
	\item A switch in the ``open'' or ``off'' position.
	\item A disconnected terminal in the circuit preventing current flow.
\end{enumerate}

\end{document}
